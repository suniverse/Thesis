\zotelo{../thesis.bib}
\newcommand{\abs}[1]{\left|{ #1}\right |}

\chapter{Wave Turbulence}
\label{app:wave-turbulence}
In the problem of wave turbulence, we are interested in the statistical properties of an ensemble of waves which interact nonlinearly.
For a real wave $\psi(x,t)$ in a $d$-dimensional box of length $L$ with periodic boundary condition, its Fourier coefficient of $\psi$ is
\begin{equation}
	\hat{\psi}(k,t) = \frac{1}{L^d}\int \md x\; \psi(x,t)e^{-i k x}.
\end{equation}
In a finite box, the wavenumbers $k_m=2\pi m/L$ with only integer $m$ allowed are discrete. 
The fact $\psi(x,t)$ is real gives $\hat{\psi}(k,t) = \hat{\psi}^*(-k,t)$.
The wavefunction can be written in terms of amplitude $J_m$ and phases $\phi_m \equiv \exp(i\varphi_m)$ with phase angle $\varphi_m$
%
\begin{equation}
	\hat{\psi}_m = \hat{\psi}(k_m,t) = \sqrt{J_m} {\phi_m}.
\end{equation}
%
We adopt the random phase approximation where all phases $\phi_m$ and amplitudes $J_m$ are independent random variables, and the phases are uniformly distributed in the unit circle $S^1$ on the complex plane.

The wave spectrum $n_m$ is defined as 
\begin{equation}
	n_m = \left(\frac{L}{2\pi}\right)^d\bracket{J_m},
\end{equation}
which is related to the two-point function of $\hat{\psi}$
\begin{equation}
	\bracket{\psi_k,\psi^*_{k'}} = n_k\delta(k-k')
\end{equation}
and the ensemble is average is take with respect to the joint probability density function of the phase and angle $P(J,\phi)$.
The four point function is
\begin{equation}
	\bracket{\hat{\psi}_1\hat{\psi}_2\hat{\psi}^*_3\hat{\psi}^*_4} = \bracket{\sqrt{J_1 J_2 J_3 J_4}}_J\bracket{\phi_1\phi_2\phi^*_3\phi^*_4}_\phi.
\end{equation}
Since $\phi_m$ is uniformly distributed on $S^1$, so $\bracket{\phi_m}=\bracket{\phi^n_m}=0$. 
In addition, each $\phi_m$ is independent, so the above average is non-zero, if all $\phi$'s can be paired with a $\phi^*$ with same $k$. 
Therefore we have
\begin{eqnarray}
	\bracket{\hat{\psi}_1\hat{\psi}_2\hat{\psi}^*_3\hat{\psi}^*_4} &=& \bracket{J_1 J_2}(\delta_3^1\delta_4^2+\delta_4^1\delta_3^2-\delta_2^1\delta_3^1\delta_4^1)\nonumber\\
	&=&	\bracket{J_1}\bracket{J_2}(\delta_3^1\delta_4^2-\delta_4^1\delta_3^2)-(\bracket{J_1^2}-2\bracket{J_1}^2)\delta_2^1\delta_3^1\delta_4^1.
\end{eqnarray}
where we ues the Kronecker delta
\begin{equation}
	\delta_2^1 = 1\; \mathrm{if}\; k_1=k_2,\; \delta_2^1 = 0\;\mathrm{otherwise}.
\end{equation}

In this chapter, we consider a simple nonlinear equation with three-wave interaction
\begin{equation}\label{wave-equation}
	\partial_t\hat{\psi_k}+i\omega_k\hat{\psi_k} = \sum_{1,2}V_{12k}\hat{\psi}_1\hat{\psi}_2 \delta_{12}^k,
\end{equation}
where $\delta_{12}^{k}$ is the Kronecker delta for $k_1+k_2$ and $k$.
\begin{equation}
	V_{12k}\equiv V(k_1+k_2-k)
\end{equation}
is the Fourier coefficient of a real function $V(x)$ at mode $k_1+k_2-k$ that characterizes the nonlinear interaction between two waves. 
Using the interaction representation variable $b_k = \hat{\psi}_k e^{i\omega_k t}/\epsilon $, Equation~\ref{wave-equation} becomes
\begin{equation}\label{wave-equation2}
	\partial_t b_k = \epsilon \sum\limits_{1,2} V_{12k} b_1 b_2 \delta_{12}^k e^{i\omega_{12}^k t},
\end{equation}
where $\omega_{12}^k = \omega_k-\omega_1-\omega_2$.

\section{Weak nonlinearity expansion}
We will seek the solution to Equation~\ref{wave-equation2} in the form of a perturbative expansion with respect to some small parameter $\epsilon$,
\begin{equation}
	b_k = b_k^{(0)}+\epsilon b_k^{(1)} +\epsilon^2 b_k^{(2)}+\cdots.
\end{equation}
Initially, only waves of the lowest order are present $b_k^{(n)}=0$ for $n>0$.

The linear time which is period of linear waves $\tau_L = 2\pi/\omega_k$ and the nonlinear timescale (which will be discussed later) is given by $\tau_{NL} = 2\pi/\epsilon^2\omega_k$.
When the nonlinearity is week, energy is transferred to higher-order modes on timescale much longer than wave periods $\tau_{NL}\gg\tau_L$.
In the regime of weak nonlinearity, we seek solutions on timescale $T$ lying in between $\tau_{L}\ll T\ll \tau_{NL}$. 
For simplicity, we can take	$T\sim 2\pi/\epsilon \omega_k$.

Expanding the Equation~\ref{wave-equation2} based on the order of $\epsilon$, we have, to the zeroth order 
\begin{equation}
	\dot{b}_k^{(0)} = 0.
\end{equation}
$b_k^{(0)} = b_k(0)$ is just the linear wave solution.

Fir the first order equation, we have
\begin{equation}
	\dot{b}_k^{(1)} = \sum\limits_{1,2}V_{12k} b_{0}^{(0)}b_2^{(0)}\delta_{12}^k e^{i\omega_{12}^k t}.
\end{equation}
The solution is given by
\begin{equation}\label{sol-1}
	b_k^{(1)} = \sum\limits_{1,2}V_{12k} b_{1}^{(0)}b_2^{(0)}\delta_{12}^k \Delta_T(\omega_{12}^{k}),
\end{equation}
with
\begin{equation}
	\Delta_T(\omega_{12}^{k}) = \int\limits_0^T e^{i\omega_{12}^k t}=\frac{e^{i\omega_{12}^k T}-1}{i\omega_{12}^k}.
\end{equation}
And the second order equation gives
\begin{equation}
	\dot{b}_k^{(2)} = 2\sum\limits_{1,2}V_{12k} b_{1}^{(1)}b_2^{(0)}\delta_{12}^k e^{i\omega_{12}^k t}.
\end{equation}
The factor $2$ to the front of right hand side comes from the symmetry of changing lower index $1\rightarrow 2$.
Substituting Equation~\ref{sol-1} for $b_1^{(1)}$, we have
\begin{equation}
	\dot{b}_k^{(2)} = 2\sum\limits_{1,2}V_{12k} b_{2}^{(0)}b_3^{(0)}b_4^{(0)}\delta_{12}^k\delta_{34}^1 \Delta_t(\omega_{34}^{1})e^{i\omega_{12}^k t}.
\end{equation}
And the solution for $b_k^{(2)}$ is
\begin{equation}\label{sol-2}
	b_k^{(2)} = 2\sum\limits_{1,2}V_{12k}V_{341} b_{2}^{(0)}b_3^{(0)}b_4^{(0)}\delta_{12}^k\delta_{34}^1 E(\omega_{34}^{1},\omega_{12}^k),
\end{equation}
with 
\begin{equation}
	E(\omega_{34}^{1},\omega_{12}^k) = \int\limits_0^T \Delta_t(\omega_{34}^{1})e^{i\omega_{12}^k t}.
\end{equation}

\section{Generating function}
%
The N-mode generating function is defined as
\begin{equation}
	\mathcal{G}(\lambda,\mu)=\bracket{\prod \exp\left[\left(\frac{L}{2\pi}\right)^d J_m\lambda_m \right] \phi_m^{\mu_m}  } = \bracket{\prod \exp\left[\left(\frac{L}{2\pi}\right)^d J_m\lambda_m \right] e^{i\mu_m\varphi_m}  }
\end{equation}
which is the Laplace transformation of $P$ in $J$ and the Fourier transformation of $P$ in $\varphi$.

The single mode generating function of the amplitude is defined as
\begin{equation}
	G_k(\Lambda_k,T)\equiv \mathcal{G}(\lambda,\mu=0)= \bracket{\exp\left[\Lambda_k |b_k(T)|^2 \right]}
\end{equation}
where $\Lambda_k \equiv\lambda_k (L/2\pi)^d$.
Expanding the generating function to order $\mathcal{O}(\epsilon^2)$, we have
\begin{eqnarray}\label{gf}
	&& G_k(\Lambda_k,T)-G_{k}(\Lambda_k,0) \nonumber\\
	&=&\epsilon \bracket{ e^{\Lambda_k J_k^{(0)}} \Lambda_k \left(b_k^{(0)}b_k^{*(1)}+b_k^{(1)}b_k^{*(0)} \right)}\nonumber\\
	&&+\epsilon^2\bracket{ e^{\Lambda_k J_k^{(0)}} \Lambda_k \left(|{b_k^{(1)}}|^2+b_k^{(0)}b_k^{*(2)}+b_k^{(2)}b_k^{*(0)}\right)}\nonumber\\
	&&+\epsilon^2\bracket{ e^{\Lambda_k J_k^{(0)}} \frac{ \Lambda_k^2}{2} \left(b_k^{(0)}b_k^{*(1)}+b_k^{(1)}b_k^{*(0)} \right)^2}.
\end{eqnarray}
The average is taken over both amplitude $J$ and phase $\phi$.
Since the amplitude and the phase are independent, we can first calculate the average over phase for all terms in the bracket after the exponential.

For the order $\epsilon$ term $\bracket{b_k^{(0)}b_k^{*(1)}+b_k^{(1)}b_k^{*(0)}}_\phi$, we have
\begin{eqnarray}
	&&\bracket{b_k^{(0)}b_k^{*(1)}+b_k^{(1)}b_k^{*(0)}}_\phi \nonumber \\
	&=&2\Re\sum\limits_{1,2}V_{12k}V^*_{34k}\bracket{b_k^{(0)}b_1^{*(0)}b_2^{*(0)}}_\phi \delta_{12}^k \Delta_T(\omega_{12}^{k})\nonumber\\
	&=&0.
\end{eqnarray}
This term vanishes because there are odd numbers of waves in the bracket to be averaged over the phase.
Hence, the difference of generating function is of the order $\mathcal{O}(\epsilon^2)$.

The first term of second order in the third line of Equation~\ref{gf} gives
\begin{eqnarray}
		\bracket{|b_k^{(1)}|^2}_\phi &=& \sum\limits_{1,2,3,4}V_{12k}V^*_{34k}\bracket{b_1^{(0)}b_2^{(0)}b_3^{*(0)}b_4^{*(0)}}_\phi \delta_{12}^k\delta_{34}^k\Delta_T(\omega_{12}^{k})\Delta^*_T(\omega_{34}^{k})\nonumber\\
		&=& 2\sum\limits_{1,2}|V_{12k}|^2 J_{1}^{(0)}J_{2}^{(0)}\delta_{12}^k|\Delta_T(\omega_{12}^{k})|^2.
\end{eqnarray}

The rest terms of second order in the third line of Equation~\ref{gf} give
\begin{eqnarray}
		&&\bracket{b_k^{(0)}b_k^{*(2)}+b_k^{(2)}b_k^{*(0)}}_\phi \\
		&=& 2 \Re \bracket{b_k^{*(0)}b_k^{(2)}}_\phi\nonumber \\
		&=&4\Re \sum\limits_{1,2,3,4}V_{12k}V_{341}\bracket{b_k^{*(0)}b_2^{(0)}b_3^{(0)}b_4^{(0)}}_\phi \delta_{12}^k\delta_{34}^1 E(\omega_{34}^{1},\omega_{12}^k)\nonumber\\
		&=&8\Re \sum\limits_{1,2}V_{12k}V_{k-21}J_k^{(0)}J_2^{(0)}\delta_{12}^k E(\omega_{k}^{12},\omega_{12}^k).
\end{eqnarray}
Using the fact that $V(x)$ is real $V_{k-21}= V^*_{12k}$ and $ \omega_{k}^{12}= -\omega_{12}^{k}$
\begin{eqnarray}
	\bracket{b_k^{(0)}b_k^{*(2)}+b_k^{(2)}b_k^{*(0)}}_\phi =8\sum\limits_{1,2}|V_{12k}|^2 J_k^{(0)}J_2^{(0)}\delta_{12}^k \Re\left[E(-\omega_{12}^k,\omega_{12}^k)\right].
\end{eqnarray}

Terms of second order in the last line of Equation~\ref{gf} give
\begin{eqnarray}
		\bracket{\left(b_k^{(0)}b_k^{*(1)}+b_k^{(1)}b_k^{*(0)}\right)^2}_\phi = 2 \Re \bracket{\left(b_k^{*(0)}b_k^{(1)}\right)^2}_\phi +2 J_k^{(0)}\bracket{|b_k^{(1)}|^2}_\phi.
\end{eqnarray}
The first part will vanish, because after inserting the solution for $b_k^{(1)}$, the average in the sum is
$\bracket{\left(b_k^{(0)}b_1^{*(0)}b_2^{*(0)}\right)^2}_\phi$. There will always be a $b_m^{(0)}$ or $b_m^{*(0)}$ pairs with itself and gives zero.

Therefore, we have the change of generating function
\begin{eqnarray}
	&& G_k(\Lambda_k,T)-G_k(\Lambda_k,0) \nonumber\\
	&=&\epsilon^2\bracket{ e^{\Lambda_k J_k^{(0)}} \left(\Lambda_k +\Lambda_k J_k^{(0)}\right)\bracket{|b_k^{(1)}|^2}_\phi}_J +\epsilon^2\bracket{ e^{\Lambda_k J_k^{(0)}} \Lambda_k \bracket{b_k^{(0)}b_k^{*(2)}+b_k^{(2)}b_k^{*(0)}}_\phi}_J\nonumber \\
	&=&2\epsilon^2 \sum\limits_{1,2}|V_{12k}|^2 \delta_{12}^k|\Delta_T(\omega_{12}^{k})|^2 \bracket{\left( \Lambda_k+\Lambda_k J_k\right)e^{\Lambda_k J_k^{(0)}}J_1 J_2}_J\nonumber\\
	&&+8\epsilon^2 \sum\limits_{1,2}|V_{12k}|^2 \delta_{12}^k \Re\left[E(-\omega_{12}^k,\omega_{12}^k)\right]\bracket{\Lambda_k e^{\Lambda_k J_k^{(0)}}J_k J_2}_J.
\end{eqnarray}

To take the average over amplitude, we assume that $k_1\neq k_2\neq k$.
Therefore the the average in the above expression can be factored as products of $\bracket{J_i}$ and $\bracket{J_k e^{\Lambda_k J_k}}$.
Using the facts that
\begin{eqnarray}
	J_k &=& \left(\frac{2\pi}{L}\right)^d n_k,\nonumber\\
	\bracket{e^{\Lambda_k J_k}} &=& G_k,\nonumber\\
	\bracket{J_k e^{\Lambda_k J_k}} &=& \frac{\partial}{\partial \Lambda_k} G_k,
\end{eqnarray}
we have
\begin{eqnarray}
	&& G_k(\Lambda_k,T)-G_k(\Lambda_k,0) \nonumber\\
	&=&\left(\frac{2\pi}{L}\right)^d 2\epsilon^2 \sum\limits_{1,2}|V_{12k}|^2 \delta_{12}^k|\Delta_T(\omega_{12}^{k})|^2 \left( \lambda_k G_k+\lambda_k \frac{\partial}{\partial \lambda_k}G_k\right) n_1 n_2\nonumber\\
	&&+\left(\frac{2\pi}{L}\right)^d 8\sum\limits_{1,2}|V_{12k}|^2 \delta_{12}^k \Re\left[E(-\omega_{12}^k,\omega_{12}^k)\right] n_2\frac{\partial}{\partial \lambda_k}G_k.
\end{eqnarray}

In the following, we are going to take two limits: 

1) the box size goes to infinity, and the discrete spectrum become continuous.
This is easily done by replacing the sum with integral and Kronecker delta with the Dirac delta.
\begin{eqnarray}
	\sum\limits_{1,2}&\rightarrow & \left(\frac{L}{2\pi}\right) ^{2d}\int\md k_1\md k_2, \nonumber\\
	\delta_{12}^k &= & \left(\frac{2\pi}{L}\right) ^{d}\delta(k_1+k_2-k).
\end{eqnarray}

2) the nonlinear parameter $\epsilon$ goes to zero.
As $\epsilon\rightarrow 0$, $T\sim 2\pi/\epsilon\omega_k\rightarrow\infty$. 
\begin{eqnarray}
	|\Delta_T(x)|^2 &=& \frac{|e^{ixT/2}-e^{-ixT/2}|^2}{x^2} = \frac{2\sin^2(xT/2)}{x^2}\rightarrow 2\pi T\delta(x),\\
	E(-x,x) &=& \int\limits_0^T \md t\;\frac{1-e^{ixt}}{-ix} = \frac{iT}{x}+\frac{1-e^{ixT}}{x^2},\\
	\Re E(-x,x) &=&\frac{2\sin^2(xT/2)}{x^2}\rightarrow \pi T \delta(x).
\end{eqnarray}

Therefore, we have
\begin{eqnarray}
	&& \frac{G_k(\Lambda_k,T)-G_k(\Lambda_k,0)}{T} \nonumber\\
	&=& 4\pi\epsilon^2  \int\md k_1\md k_2\;\abs{V_{12k}}^2 \delta_{12}^k\delta(\omega_{12}^{k}) \left( \lambda_k G_k+\lambda_k \frac{\partial}{\partial \lambda_k}G_k\right) n_1 n_2\nonumber\\
	&&+ 8\pi\epsilon^2\int\md k_1\md k_2\;\abs{V_{12k}}^2 \delta_{12}^k \delta(\omega_{12}^{k}) n_2\frac{\partial}{\partial \lambda_k}G_k.
\end{eqnarray}
We can replace the first line with the time derivative $\dot{G_k}$ to get time evolution of $G_k$.
\begin{equation}
	\dot{G}_k = \lambda_k \eta_k G_k+(\lambda_k^2 \eta_k + \lambda_k\gamma_k)\frac{\partial}{\partial \lambda_k}G_k,
\end{equation}
with
\begin{equation}
	\eta_k = 4\pi\epsilon^2  \int\md k_1\md k_2\;\abs{V_{12k}}^2 \delta_{12}^k\delta(\omega_{12}^{k}) n_1 n_2
\end{equation}
and 
\begin{equation}
	\gamma_k = 8\pi\epsilon^2\int\md k_1\md k_2\;\abs{V_{12k}}^2 \delta_{12}^k \delta(\omega_{12}^{k})n_2.
\end{equation}

\section{Kinetic equation}

The inverse Laplace transform of $G_k$ gives the probability distribution function $P(J)$
\begin{equation}
	P(J_k) = \int\limits_{\zeta-i\infty}^{\zeta+i\infty}\md \lambda G(\lambda)e^{-\lambda_k J_k}.
\end{equation}
Therefore,
\begin{equation}
	\dot{P}(J_k) =\frac{\partial}{\partial J_k}\left( J_k \eta_k\frac{\partial}{\partial J_k}P + \gamma_k J_k P \right).
\end{equation}
From the definition, wave spectrum is the first moment of the probability distribution function $P(J)$
\begin{equation}
	n_k = \left(\frac{L}{2\pi} \right)^d\int \md J_k \; J_k P(J_k).
\end{equation}
The time evolution of wave spectrum is
\begin{eqnarray}\label{kinetic}
	\dot{n}_k & = & \left(\frac{L}{2\pi}\right)^d\int \md J_k \; J_k \frac{\partial}{\partial J_k} \left( J_k \eta_k\frac{\partial}{\partial J_k}P + \gamma_k J_k P \right)\nonumber\\
	&=& - \left(\frac{L}{2\pi} \right)^d\int \md J_k\; \left( J_k \eta_k\frac{\partial}{\partial J_k}P + \gamma_k J_k P \right)\nonumber\\
	&=&\eta_k-\gamma_k n_k.
\end{eqnarray}

Equation~\ref{kinetic} is called the kinetic equation, it describes the evolution of wave spectrum due to nonlinear interactions.
We can also now check the nonlinear time scale $\tau_{NL}$. 
The time evolution for $\dot{n}_k$ starts from order $\epsilon^2$. 
The nonlinearity induces significant changes to the wave spectrum on the time scale $1/\epsilon^2$, which justifies our use of nonlinear time scale.
In the weak nonlinearity limit, the change of wave spectrum is completely due to the resonant three-wave interactions where $\omega_{12}^k= \omega_k-\omega_1-\omega_2=0$. 
In most cases, energy spectrum is proportional to the wave spectrum, so energy also transfers only through resonant interactions.
Non-resonant interaction leads to fast oscillation on the short time scale. 
Actually, the non-resonant interaction dominates the wave evolution on the short time scale. 
However, this fast oscillation does not accumulate, and the net effect will be averaged to zero after long time. 
We neglect the oscillation on the short scale, but focus on the long time evolution of the system that we compare the difference of $G_k$ on the long time scale $T\ll\tau_L$.
Resonant interactions, on the other hand, accumulates. 
Changes of wave spectrum on the time scale much longer than the wave period are purely due to resonant interactions.

The integration of resonant interactions involves a delta function $\delta(\omega_{12}^k)$. However, nonlinearity can broaden the resonance line in the $k$ space. The reason is that true dispersion relation $\omega(k) = \omega_L(k) + \Delta \omega$, is different from the linear one. 
The modification term can depend on other factors, e.g. the amplitude of interacting waves. Resonant interactions can be satisfied when $\abs{\omega_{12}^k}\leq \Gamma$, corresponding to a finite width $\Gamma\sim \gamma_k$ in the frequency space.




% Local Variables:
% TeX-master: "../thesis"
% zotero-collection: #("16" 0 2 (name "Thesis"))
% End:
