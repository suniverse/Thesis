\begin{center}

{\Large \bf Abstract} \vskip.2in

{\Large \bf \thesistitle} \vskip.2in

{\Large  \thesisauthor} \vskip.2in

\end{center}

The ultra-strong magnetic fields of magnetars have profound implications for their radiative phenomena. 
We study the dynamics of strong magnetic fields inside and outside magnetars. 
Inside the magnetar, the strong magnetic stress can break the crust and trigger plastic failures. 
The interaction between magnetic fields and plastic failures is studied in two scenarios: 
1. Internal Hall waves launched from the core-crust interface can initiate plastic failures and lead to X-ray outbursts. 
2. External Alfven waves produced by giant flares can also initiate crustal plastic failures which dissipate the waves and give rise to delayed thermal afterglow. 
The crustal dissipation of Alfven waves is competed by the magnetospheric dissipation outside the magnetar. 
Using a high order simulation of Force-Free Electrodynamics (FFE), we found that the magnetospheric dissipation of Alfven waves is generally slow and most wave energy will dissipate inside the magnetar.

% Local Variables:
% TeX-master: "../thesis"
% End: