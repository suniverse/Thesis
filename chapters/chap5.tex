\zotelo{../thesis.bib}

\chapter{Conclusions}
\label{chap:conclusions}

In this dissertation, we explore several topics regarding the magnetodynamics inside and outside magnetars. 
We focus on the dynamics of magnetic fields coupled with the mechanical response of the magnetar, especially the plastic deformation of the crustal materials. 
In the magnetar crust, we have shown that the Hall evolution of strong magnetic fields can accumulate magnetic stress which will trigger plastic failures.
Motion of magnetic footpoints on the magnetar surface is induced as a result of plastic crustal deformations which twists the magnetosphere.
Subsequent X-ray emission from the untwisting of the twisted magnetosphere are demonstrated to be consistent with observations of magnetar outbursts in our 1D simulations.
The reverse process of magnetic energy released in the magnetosphere to be transported into the crust can also occur, especially when the energy is carried by \alfven waves during giant flares.
We have calculated the transmission coefficient of \alfven waves into the crust is of order $\sim 10\%$.
The transmitted waves are compressed in the crust with their strain increased.
Eventually, plastic failures are triggered by the waves and dissipate the magnetic energy to heat in the crust.
The heated crust loses the major fraction of heat through neutrino emission while a small fraction is conducted to the surface and powers a thermal afterglow.
The process of crustal dissipation of \alfven waves competes with the magnetospheric dissipation through the turbulent cascade.
We have performed numerical simulations in the setting of force-free electrodynamics to study the turbulent dissipation process.
Our results suggest that the turbulent dissipation is slow compared to the crustal dissipation unless the waves have amplitude much larger than the background field.
Conversion of \alfven pairs to fast waves which can escape the magnetosphere provides another channel of wave energy loss.
This process is also present in our simulations and is also found to be slow.
The breakdown of force-free conditions $E<B$ and $\be\cdot\bb=0$ is thought to be an efficient dissipation channel of magnetic energy.
In our simulations, this process is observed for \alfven waves with anti-aligned magnetic field.
However, through comparison with relativisitic MHD simulations, we found that energy dissipation is overestimated in FFE simulations and requires further study using kinetic simulations.

The dynamics of strong magnetic fields have profound implications for the high energy radiation from the magnetars.
In this thesis, the interaction of magnetic fields and the magnetar crust is proposed to produce magnetar outbursts as well as delayed afterglow of giant flares.
But this thesis is far from a complete study of the effects of strong magnetic fields in the crust or in the magnetosphere.
The outburst model from crustal failures occurs on the timescale much longer than a millisecond.
How magnetic energy can be quickly released into the magnetosphere on the millisecond scale necessary to explain the short rise time of magnetar bursts and giant flares is still not clear.
Magnetic reconnection as a fast dissipation mechanism of magnetic energy is not studied in this dissertation.
Whether magnetic reconnection can be triggered in 3D simulations and explain magnetar bursts and giant flares remains to be explored.
The high surface temperature of magnetars poses another unanswered theoretical question for the heating and cooling of magnetars.
In addition, the heating of magnetar crust from plastic deformations can thermally unpin the superfluid vortices and the redistribution of vortices can lead to interesting timing anomalies \citep{1996ApJ...457..844L}.

New high energy astrophysical phenomena and future observational missions also require better knowledge of the dynamics of strong magnetic fields.
Magnetic instabilities and dynamos are observed in simulations of binary neutron star mergers to amplify the magnetic fields to $10^{16-17}$~G \citep{2014PhRvD..90d1502K,2015ApJ...809...39G}.
High energy radiation from the strong magnetic fields might serve as another electromagnetic counterpart of binary neutron star merger events.
Fast radio bursts (FRB) are another contemporary mystery of theoretical high energy astrophysics, and magnetars are the central source in many theoretical proposals \citep{2017MNRAS.468.2726K,2018ApJ...868L...4M,2019MNRAS.485.4091M}.
The physics of the strong magnetic fields is essential to connect existing FRB models with observations in order to prove or falsify them.
Future X-ray observations, especially IXPE \citep{WEISSKOPF20161179} and eXTP \citep{2016SPIE.9905E..1QZ}, are proposed to measure the polarization of X-ray emissions which can probe the magnetic field structure of magnetars and possibly QED effects in the ultra-strong magnetic fields.
The study of the magnetodynamics inside and outside magnetars remains interesting and exciting in the future.





% Local Variables:
% TeX-master: "../thesis"
% zotero-collection: #("16" 0 2 (name "Thesis"))
% End: